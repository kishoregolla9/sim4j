Applications of wireless sensor network (WSN) often expect knowledge of the precise location of the nodes. One class of localization protocols patches together relative-coordinate, local maps into a global-coordinate map.  These protocols require some nodes that know their absolute coordinates, called anchor nodes.  While many factors influence the node position errors, in this class of protocols, using Procrustes Analysis, the placement of the anchor nodes can significantly impact the error.  Through simulation, using the Curvilinear Component Analysis (CCA-MAP) protocol, we show the impact of anchor node placement and propose a set of guidelines to ensure the best possible outcome, while using the smallest number of anchor nodes possible.  Scientists and researchers using sensor networks are thus enabled to focus on the sensed data with confidence in the node localization results.
