Applications of wireless sensor network (WSN) often expect knowledge of the precise location of the nodes. One class of localization protocols patches together relative-coordinate, local maps into a global-coordinate map.  These protocols require nodes that know their absolute coordinates, called anchor nodes.  While many factors influence the calculated position errors, in this class of protocols, the placement of these anchor nodes significantly impacts the error.  Through simulation, using the Curvilinear Component Analysis (CCA-MAP) protocol, we show the impact of anchor node placement and a set of rules to ensure the best possible outcome.  Scientists are thus enabled to focus on the sensed data, and rely on a maximum node position error.
