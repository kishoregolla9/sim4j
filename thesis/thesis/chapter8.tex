\chapter{Conclusion}

While the simulations in this study uses the CCA-MAP algorithm, the results apply to any protocol in which a local coordinate system is transformed into a global coordinate system using a set of anchor nodes, like MDS-MAP.  Further, the scope of the study extends beyond wireless sensor network protocols and can be applied to any transformation problem where a small subset of points is used as the basis to transform a set of points.

This study provides two key recommendations for network designers when placing anchor nodes.  Namely, make sure that the sum of the distance between anchor nodes is at least ten times the radio range and that the minimum height of the triangle formed by the anchor nodes is at least equal to the radio range.  Further, the larger these two metrics are, the lower the mean location error of the network will be and the lower the probability of using an anchor set that will cause extremely poor localization performance.  We have further shown that these criteria apply to network topologies where the overall network area is two-dimensional, but fails when the network topology is extremely narrow compared with the radio range, as in pipelines or roads.

