\chapter{Related Work on Anchor Node Placement}

While much attention has been paid to localization accuracy and computational effort, anchor node placement is often recognized, but dismissed as future study.   

\section{Empirical Evidence}

Often, authors will come across anchor placement by accident and discuss it based on their own empirical evidence.  Shang, et al.\cite[p. 964]{MDS-MAP} and Li, et al.\cite[p. 11]{CCA-MAP07} both choose anchors at random within the network.  Although, Shang does mention that a co-linear set of anchors chosen in one example "represents a rather unlucky selection", without supporting evidence of why this is unlucky.

Earlier work by Doherty, et al.\cite{Doherty} requires anchor nodes to be placed at the edges, and ideally at the corners of the network.  In this case, however, the algorithm is a simple constraint problem.  One constraint requires that all the unknown nodes be places within the convex hull of the anchors, and therefore, better results are obtained when anchors are at the corners.

\section{Explicit Studies of Anchor Node Placement}
While few, there have been a few explicit studies of anchor node placement.  Hara, et al.\cite{Hara} propose a method of choosing anchor node locations to achieve a specific accuracy target.  The proposal, however, only applies to rectangular network areas and that anchor nodes must be placed at the center of a sub-rectangle of the original rectangle when divided into equal sized rectangles.  Further, it assumes simple RSSI-based localization.  Other studies also focus on the effect of indoor conditions and anchor placement as it relates to RSSI and other radio propagation measurements\cite{Zemek}.

Ash, et al.\cite{Ash} provide analytical proof that placing anchor nodes uniformly around the perimeter of a network proves the best results, in the absence of any other information about the sensor node positions. However, again this assumes a rectangular network, and more importantly a simple localization algorithm like \cite{Doherty} or other simple triangulation techniques.  When using all inter-node distances at once, as in MDS-MAP and CCA-MAP, this analysis breaks down.

TODO get book Protocols and Architectures for Wireless Sensor Networks By Holger Karl, Andreas Willig
https://catalogue.library.carleton.ca/record=b2282088~S9
