\chapter{Conclusion}

\section{Summary and Contribution}
This study provides two key guidelines for network designers and users of wireless sensor networks when choosing anchor node positions or assessing the quality of localization results.  Namely, make sure that the sum of the distance between anchor nodes is at least ten times the radio range and that the minimum height of the triangle formed by the anchor nodes is at least equal to the radio range.  Further, the larger these two metrics are, the lower the mean location error of the network will be, on average, and the lower the probability of using an anchor set that will cause extremely poor localization performance.  Effectively, this means do not put the anchor nodes in a straight line or close to each other.  We have further shown that these criteria apply to network topologies where the overall network area is two-dimensional, but fails when the network topology approaches one-dimension, meaning it is extremely narrow compared with the radio range, for example when monitoring pipelines or roads.

While the simulations in this study use the CCA-MAP algorithm, the results apply to any protocol in which a local coordinate system is transformed into a global coordinate system using a set of anchor nodes with Procrustes analysis.  Further, the scope of the study extends beyond wireless sensor network protocols and can be applied to any transformation problem where a small subset of points is used as the basis to transform a set of points.

\section{Future Work}
While we have provided some recommendations for anchor placement, there are a few key areas where future study could enhance the results.  First, an examination of the other factors affecting localization performance would be beneficial, as described in Chapter~\ref{sec:otherfactors}.  Specifically, analysis of the results presented here, as they are affected by network connectivity levels. A raw examination of other as of yet undiscovered factors could also uncover ways to further improve localization accuracy.  Second, expanding these results to three-dimensions may benefit some network designers.  For example, a sensor network may be deployed through a high-rise building or on a bridge.  While we expect that results to be similar, it is worthwhile to confirm.  Outside the scope of sensor networks and the physical world, these results could also apply to any dimension of data.  This would effectively be an exhaustive study of Procrustes analysis where a set of data in one coordinate system is transformed into another coordinate system based on a subset of known points.  Last, finding other methodologies besides Procrustes analysis to provide a transformation between local and global coordinates.  This may be especially beneficial if applied to pipeline topologies specifically.